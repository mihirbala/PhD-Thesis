In the previous chapters, I have shown that SteelEagle is an effective solution to the onboard compute tradeoff. By taking advantage of edge computing, it can induce full autonomy on lightweight COTS drones without demanding heavy onboard compute. I showed how to connect COTS drones to the edge using off-the-shelf relays, and how to design an edge backend that could support autonomous operation. I also evaluated the system on a set of benchmarks to show its capabilities. Lastly, I demonstrated how SteelEagle could be easily ported to new drone hardware. In this chapter, I will summarize the major contributions of this dissertation and suggest future research directions.

\section{Summary of Contributions}

\subsection{Active Vision with Ultralight Drones}
Current autonomous drones have seen limited use in urban settings because their heavy weight exceeds government regulatory limits. SteelEagle presents a way to achieve full autonomy on lightweight drones within these limits by offloading computation normally done on heavy onboard resources to a nearby cloudlet via a cellular connection. I demonstrated how such a connection can be established on consumer drones and I showed that an initial proof-of-concept system was able to execute active vision tasks. My lightest prototype had a takeoff weight of just 360~g, only 110~g over the FAA regulatory cutoff, which is much closer than traditional autonomous drones that typically weigh over 1~kg. I expect that future improvements in drone and mobile hardware may drive this number down considerably.

\subsection{Autonomy on COTS Hardware}
Most drone research tends to focus on custom-built hardware that is tailor made for its intended task. SteelEagle, by contrast, is designed to work with commercial off-the-shelf (COTS) components to promote easy deployment. Critically, my prototypes show that it is possible to induce full autonomy on COTS drones that have no cellular connection or compute by mounting a lightweight COTS relay device on board which acts as an intermediary between the drone and the cloudlet. To my knowledge, this has never been done before. This control paradigm has the potential to be disruptive, possibly introducing a new market segment of lightweight, COTS edge-based UAVs. 

\subsection{An Open-Source Edge Pipeline for Drone Intelligence}
While edge-based drone intelligence is not new, many solutions either have a narrow range of compatibility or are closed-source. SteelEagle represents the first open-source edge intelligence framework that is designed to support a wide range of AI backends. Its cognitive engine approach, borrowed from Gabriel~\cite{Ha2014}, creates a plug-and-play development environment which can easily integrate new AI models into the processing pipeline and ride the wave of AI innovation in algorithms and hardware accelerators. 

\subsection{Benchmarks for Autonomous Drone Agility}
Existing benchmarks for drone agility often leverage simulated environments to encourage accessibility and reproducibility. However, in my experience, simulations struggle to capture the flight dynamics of real aircraft. My benchmark suite attempts to find a middle ground by providing a reproducible and accessible setup while testing drones in real flight. To the best of my knowledge, it is the first such benchmark suite of its kind. Over time, more benchmarks can be added to the suite and scoring can be refined. Additionally, as more drones are incorporated into SteelEagle, each can be scored on the suite. The benchmarks are parameterized so that future versions can easily increase in difficulty as drone technology improves. 

\subsection{Hardware Agnostic Operations}
Today, the drone space is fragmented with dozens of custom SDKs and autopilots, each used for a specific set of aircraft. Without a unified development environment, it is impossible to port code written for one class of drone to another. SteelEagle attempts to solve this by introducing SteelEagle OS, an operating system designed to integrate varied drone SDKs under a single unified API. The architecture of SteelEagle OS is also built to support as wide a range of control schemes as possible, including those that include a local RC controller. This system, if seen through to its full potential, could be the primary development tool for mission-centric drone programming within the research community and in industry practice.

\section{Future Work}

\subsection{Other Robotic Platforms}
Quadcopters are not the only types of robotic platforms that can benefit from edge computing. Rovers, fixed-wing aircraft, helicopters, sea vehicles, and more could all leverage edge computing in a similar way to SteelEagle drones. MAVLink, the ubiquitous UAV control protocol, is actually designed to work with many of these platforms, and sees actual use in these settings today. SteelEagle could replicate the success of MAVLink, and could open the door for multi-domain robotic cooperation between compatible systems.

\subsection{Drone Swarms}
The majority of this dissertation is concerned with testing the performance of a single drone. However, most drone research is now headed towards drone swarms. With drone swarms, dozens of aircraft, much like a swarm of bees, collaborate towards a shared goal. The idea is that such aircraft would be cheaper and less capable than larger platforms, but would perform equivalently in aggregate. This perfectly fits the ethos of SteelEagle which seeks to connect small, cheap drones to powerful compute without increasing cost or weight. Thus the logical next step is to make drone swarms a first-class citizen of SteelEagle. In the future, this could mean heterogeneous swarms of cheap, insect-like aircraft coordinated by one or several cloudlets for building inspection, surveillance, or weather monitoring.

\subsection{Expressive Mission Specifications}
In order to achieve collaboration between swarm aircraft, changes must be made to the SteelEagle mission specification. Currently, it is a static script that the drone follows until completion. While this can support some dynamic action, it lacks the structure to cleanly integrate collaborative behavior. One option is to shift from a static flight script to a finite state machine, where behaviors are linked together by a pre-defined transition function. This could eventually turn into an independent domain-specific language which, when integrated with the SteelEagle API, could be a cross-platform coding tool that spans many drone control paradigms.

\subsection{On-Demand Edge Deployment}
For real flight operations, especially in adversarial environments, it is rarely the case that compute resources are pre-provisioned for missions. Instead, they must be allocated on demand to support the current task. In its present configuration, the SteelEagle backend has no mechanism for dynamic deployment; it must be set up on a machine manually and then drones must be notified of its address within the network for the system to function. In the future, this startup routine could become more dynamic. For example, when a drone requests computation, a central authority could find the closest cloudlet and provision a SteelEagle backend for the drone. It could also respond to heightened demand through server replication, or migrate cloudlet sessions across servers as drones are in flight~\cite{Ha2013}.

\subsection{Leveraging Onboard Compute}
As shown by Section~\ref{sec:djiminipro}, onboard compute can sometimes outperform edge-based solutions. In these situations, the SteelEagle OS local compute driver is responsible for taking over computation duties from the remote compute driver. However, there are times when the separation between what should be done onboard versus on the cloudlet is not cleanly separated. In particular, prior work has shown the benefits of a collaborative approach in which local computation informs when data is promising enough to be shipped to the cloudlet for further analysis~\cite{Iyengar2023,Wang2017}. This approach, known as ``offload shaping'' can save bandwidth and energy. Currently, SteelEagle OS cannot use offload shaping, but it can be extended to support it. 

\subsection{Transient Disconnected Operation}
A major weakness of SteelEagle is its dependence on its connection to the edge. If network disconnection occurs, SteelEagle OS is designed to safely return the drone to its launch point. This is the only possible course of action for thin client drones. Even so, for more capable clients, limited disconnected operation could be possible. One can imagine a drone with marginal onboard compute that could try to seek out the last location where it had service, or alternatively continue its mission with degraded performance until reconnection occurs. Such functionality could be critical for real world deployments of SteelEagle, especially in adversarial settings, where consistent connectivity cannot be expected.

\section{Closing Thoughts}
SteelEagle presents the first open-source attempt at bringing full autonomy to lightweight, COTS drone platforms. It promotes portability with its hardware-agnostic modular design, and is positioned well to take advantage of future research. My hope is that its accessibility could make it a ``Linux for drones'', the default platform that drone autonomy developers the world over deploy their projects on. Such democratization could herald a new era of drone innovation, unbound by the constraints of proprietary SDKs while preserving public safety consistent with existing government regulation.




