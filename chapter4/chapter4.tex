Fully-autonomous drones are only as effective as the artificial intelligence software and decision-making protocols they use. SteelEagle is no different; its performance is tightly correlated with the sensor processing and control algorithms running on the edge. However, its offload-based design poses additional constraints: all sensor data from the drone is inherently latent and may have suboptimal throughput. Furthermore, cloudlets may be required to serve multiple drones, and may have to deal with changing network conditions that hamper bandwidth.

In this chapter, I will discuss my architecture for the SteelEagle backend. I will show how it addresses these unique problems and how it adapts to its network environment. In Section~\ref{sec:remote-intelligence-framework}, I outline the overall design of the system and its interaction with drone clients. In Section~\ref{sec:eval}, I evaluate the system on a few basic tasks to understand its capabilities.

\section{An Edge Intelligence Framework for Drones}
\label{sec:remote-intelligence-framework}
The SteelEagle backend is made up of three main modules: compute, command and control, and storage. To handle transmission between these modules and between the cloudlet and external clients, traffic is split into two channels, the data plane and the control plane. Each channel uses a different communication protocol to suit their specific quality-of-service demands. Figure~\ref{fig:sys-arch} shows the full system architecture. The overall system is characterized by two main data flows: the remote processing flow through the data plane and the remote control flow through the control plane. I will outline both to illustrate how the system operates.

\begin{figure}
    \centering
    \includegraphics[width=0.9\linewidth]{chapter4/FIGS/arch.png}
    \begin{captext}
    \small The remote processing flow, which handles the processing of the drone's sensor stream and the proliferation of generated results, is highlighted in blue. The remote control flow, which delivers all commander messages and auto-generated commands from the command and control module, is highlighted in red.
    \end{captext}
    \caption{SteelEagle System Architecture}
    \label{fig:sys-arch}
\end{figure}

\begin{figure}
    \centering
    \includegraphics[width=0.8\linewidth]{chapter4/FIGS/gabriel.png}
    \begin{captext}
    \\[0.1cm]
    \small In the case of SteelEagle, the Wearable Device is replaced with a drone communicating over cellular. The User Guidance VM not only sends processing results back to the drone through the data server (as shown in Figure~\ref{fig:sys-arch}), but also stores results in the storage module.
    \end{captext}
    \caption{Gabirel Cognitive Assistance Model (Adapted from ~\cite{Ha2014})}
    \label{fig:gabriel-cognitive-assistance}
\end{figure}


\subsection{Remote Processing Flow}
Perhaps the most important data flow in SteelEagle is the remote processing flow, outlined in blue in Figure~\ref{fig:sys-arch}. The role of this flow is straightforward: retrieve sensor data and telemetry from the drone, run the requested algorithm by the current mission, return results to relevant clients, repeat. Yet there are additional considerations that complicate this seemingly simple picture. First, bandwidth is a precious resource for mobile systems, and availability may vary highly based on current network conditions~\cite{Forman1994}. For this reason, it is important that the stream retrieval portion of the data flow can cope with bandwidth constriction without introducing too much latency. Second, many DNNs require vast computational resources to run quickly. Some may not be able to sustain sufficient throughput to keep up with the drone's stream. If the data flow takes on multiple tenants without adapting its inference rate to model throughput, it could lead to queueing delay and thus added latency.

\subsubsection{Gabriel and the Cognitive Assistance Model}
To address both of these problems, the SteelEagle remote processing flow leverages the Gabriel Cognitive Assistance model~\cite{Ha2014}. Proposed in 2014, Gabriel is a bandwidth-adaptive edge intelligence system that provides near real-time sensor processing results to mobile clients (Figure~\ref{fig:gabriel-cognitive-assistance}). Gabriel is bandwidth-adaptive and model-throughput-adaptive thanks to its token-based flow control protocol. Token-based flow control is a technique for avoiding queueing delay in a client-producer server-consumer system using data objects called tokens. Tokens are a certificate that signal the server or client to act; if the server has a token, it has client data that needs to be processed and if the client has a token, it must send its most recently produced data to the server. 

At system start, a set number of tokens is agreed upon between the server (also called the control VM in Figure~\ref{fig:gabriel-cognitive-assistance}) and client. The server and client exchange these tokens based on their individual processing rates. For instance, if the client's send rate outpaces the server's processing rate, the server will be in possession of a large share of the tokens and vice versa. If no tokens remain on the client side, the client drops the current payload until it sees a new token. If no tokens remain on the server side, it waits to receive new client data. In this model, the client only ever sends the most recently produced payload which prevents the server from receiving highly latent data.

In Gabriel, each distinct sensor processing algorithm is called a ``cognitive engine''. Gabriel supports running several cognitive engines simultaneously, and manages their input data queues. Each cognitive engine is run in a separate container on the host machine, and is shipped data via an inter-process communication channel. Gabriel's tokens are shared among the engines which effectively bottlenecks the client send rate to the throughput of the fastest engine. On slower engines, queued data is dropped to ensure the latest data available is processed.

\subsubsection{SteelEagle Upgrades to Gabriel}
Gabriel is designed for mobile clients that have stable connections to the edge over WiFi. Due to this, underlying connections are made via websockets. Websockets are a point-to-point connection medium that are intended for web browser to server links. They are performant but must be modified to properly deal with disconnections. In the case of SteelEagle,  cellular links to the backend can be easily disrupted by drone motion or interference. This renders websockets sub-optimal. To fix this, SteelEagle replaces the underlying communication in Gabriel with ZeroMQ sockets that provide auto-reconnection among other quality-of-service guarantees~\cite{ZeroMQ}. Other connections within SteelEagle that are not Gabriel-regulated use pure ZeroMQ sockets (Figure~\ref{fig:sys-arch}).

Within SteelEagle, the Gabriel control VM is called the \textit{data server} and cognitive engines are managed by an entity called the \textit{compute module} (see Figure~\ref{fig:sys-arch}). In most cases, the compute module will spawn a pre-specified set of algorithms demanded by a drone mission at mission start. In the future, I envision that it may also support runtime addition and deletion of cognitive engines and dynamic scale-out at runtime to adapt to constrained computation resources or new mission parameters.

Unlike the default Gabriel implementation, SteelEagle supports multiple consumer clients of processing results (generated by the User Guidance VM in Figure~\ref{fig:gabriel-cognitive-assistance}). The main consumer of these results other than the drone is the storage module (see Figure~\ref{fig:sys-arch}). The storage module is a database responsible for logging all processed results in addition to the raw sensor and telemetry sent by the drone. This can be read by other monitoring clients such as the command and control module if needed for orchestration purposes.

\subsubsection{Supported Cognitive Engines}
As a result of its Gabriel-based design, SteelEagle supports a wide range of cognitive engines. To add new engines, users simply need to work within the Gabriel cognitive engine interface and connect to the Gabriel server~\cite{Ha2014}. The server will automatically configure the requested type of sensor stream and manage timely delivery of data to the engine. For initial evaluation purposes, I have configured two engines useful for drone operations: object detection and obstacle avoidance. I will detail their implementation in Section~\ref{sec:eval}.

\subsection{Remote Control Flow}
The remote control flow handles interactions between humans and SteelEagle drones. Within the SteelEagle ecosystem, since the drones are fully-autonomous, there are no conventional human pilots. Instead, humans who interact with SteelEagle drones are referred to as ``commanders''. Commanders create missions, send missions to aircraft, and monitor telemetry data to ensure continued safe operation. They also have the ability to manually pilot a connected aircraft in case of emergency. Figure~\ref{fig:commander} shows the commander UI. It provides continuous feedback on the progress of the mission on a map.  It also displays live sensor streams, including video, transmitted by the drone. It includes several buttons to upload missions, take manual control of one or more aircraft, and order one or more aircraft to return home.

Messages sent by commanders are aggregated in the command and control module (see Figure~\ref{fig:sys-arch}). Here, messages are relayed to the control server which then sends them to the target aircraft over a direct, unregulated socket connection. This is in contrast to the data server, which uses token-based flow control socket connections. The reason for this difference is that messages sent by commanders, especially emergency commands, are deemed to be high priority and must therefore be forced over the link as quickly as possible without any regard for bandwidth.

The command and control module also acts as an air traffic controller. That is, it manages the airspace inhabited by connected aircraft to ensure there are no mid-air collisions. For instance, it allocates non-intersecting altitude slices to different drones operating in the same mission area to ensure they are flying at different altitudes. The command and control module maintains up-to-date telemetry from all connected drone clients via the storage module and can therefore detect potential crashes before they happen.

\begin{figure}
    \centering
    \includegraphics[width=1.0\linewidth]{chapter4/FIGS/commander.png}
    \begin{captext}
    \small \textit{\textbf{1.} Active drones with displayed battery percentage, altitude, and GPS location, \textbf{2.} Selection of command recipients, \textbf{3.} Mission upload tool, \textbf{4.} Emergency and manual command buttons}
    \end{captext}
    \caption{Commander Interface}
    \label{fig:commander}
\end{figure}

\subsubsection{Drone Missions}
Prior to flight, a mission is planned using a toolchain that starts with Google MyMaps. The flight route can be specified using waypoints, line segments and polygons. Actions such as capturing images, tracking particular objects, or avoiding obstacles can be associated with parts of the route, and can be linked together to form a primitive behavior tree. Behavior trees are a common way of expressing missions in a variety of robotics settings~\cite{Ghouzli2023}. An offline compilation step, usually performed on a commander computer, transforms the high-level specification into low-level drone-specific runtime actions on the cloudlet and on the drone.  In the case of the Parrot Anafi-Galaxy Watch 4 prototype, the compiler output is expressed via Ground SDK API Java classes and packaged into an Android DEX file. A drone simulator can be used for testing and visualization of this output. 

In its current form, SteelEagle assumes a one-to-one correspondence of drones and missions. In particular, each drone executes its given mission in isolation, without cooperation. In future, to support swarm operations, this would change. For now, this simple abstraction provides enough for basic autonomous drone operations.


\section{Evaluation}
\label{sec:eval}

Given the prototype constraints discussed in Section~\ref{sec:austere-computing} and my proposed backend architecture, I ask the following question: 
{\em ``Can my flight platform, with severe constraints on both local compute and edge offload, achieve autonomy for active vision?''} Recall, active vision tasks are those that require the drone to react in real time to its surroundings~\cite{Aloimonos1988, Ognibene2013}. To answer this question, I conduct experiments with a series of tasks of increasing difficulty that probe and quantify the limits of my flight platform:

\begin{itemize}

\item{{\bf Task-1:} Detecting a moving object while hovering.}

\item{{\bf Task-2:} Detecting and tracking an object by yawing to keep it in
    the field of view (FOV) as it moves.}

\item{{\bf Task-3:} Detecting and tracking an object by following it at a
    fixed leash distance as the object moves.}

\item{{\bf Task-4:} Detecting a moving object from high altitude, and
    then descending to closely inspect it.  }

\item{{\bf Task-5:} Detecting and avoiding static obstacles.}

\end{itemize}

My goal is to perform these tasks from takeoff to landing with no
human intervention.  Where possible, I test my
system against the Parrot Anafi Ai, a semi-autonomous COTS drone with LTE support, limited onboard tracking and obstacle avoidance.  It weighs more
than twice as much as my flight platform, and costs over five times as much~(Figure~\ref{fig:featurematrix}).  A less-constrained platform
can clearly do more, but it may weigh more too.  My focus is on
whether my 360~g drone-watch platform is too weak or just enough.

The cloudlet used in the following experiments has 36 CPU cores, 128GB of RAM and an NVIDIA GeForce GTX 1080
Ti GPU.  It is capable of using a private CBRS LTE network for low end-to-end latency. However, since the Galaxy Watch is not able to connect over CBRS, all presented results are based on public cellular network infrastructure.

\subsection{Flight Duration}
\label{sec:flightduration}

Since my platform consists of mounting external components on an
existing drone, a reduction in flight duration is expected. This
reduction is important to quantify, as it directly impacts my
platform's range and practicality. In Figure~\ref{fig:battery}, I
show the hover time of the Parrot Anafi with a 0~g, 40~g, and 60~g
payload weight. The Galaxy Watch with harness is around a 40~g
payload.

\begin{table}
	\centering
	\begin{tabular}{|l|c|c|c|}
		\hline
		& My Platform & Base Parrot Anafi & Parrot Anafi Ai \\
		\hline
		Cost & \$769 & \$469 & \$4,500 \\
		\hline
		Weight & 360~g & 320~g & 898~g \\ 
		\hline
		Detection & \cellcolor{green!30}Yes & \cellcolor{red!30}No & \cellcolor{red!30}No \\
		\hline
		Tracking & \cellcolor{green!30}Yes & \cellcolor{yellow!30}Yes* & \cellcolor{yellow!30}Yes*\\
		\hline
		Avoidance & \cellcolor{green!30}Yes & \cellcolor{red!30}No & \cellcolor{green!30}Yes\\
		\hline
		Programmable & \cellcolor{green!30}Yes & \cellcolor{green!30}Yes & \cellcolor{green!30}Yes\\
		\hline
		4G/5G & \cellcolor{green!30}Yes & \cellcolor{red!30}No & \cellcolor{green!30}Yes\\
		\hline
	\end{tabular}
	\begin{captext}
            \\[0.1cm]
		\centering
		\small *~Requires assistance from the pilot for initial object detection.
	\end{captext}
	\caption{Flight Platforms Relevant to my Experiments}
	\label{fig:featurematrix}
\end{table}

\begin{table}
\centering
\begin{tabular}{|l|c|c|c|c|c|c|}
\hline
&0~g &40~g &\cellcolor[HTML]{FF9470}\% Reduction 
 &60~g &\cellcolor[HTML]{FF9470}\% Reduction \\
 \hline
Battery 1 & 21:04 & 18:43 &11.16\% & 17:18 &17.88\% \\
 \hline
Battery 2 & 22:54 & 19:35 &14.48\% & 18:57 &17.25\% \\
 \hline
Battery 3 & 17:57 & 14:23 &19.87\% & 13:00 &27.58\% \\
 \hline
Battery 4 & 20:00 & 16:43 &16.42\% & 15:12 &24.00\% \\
 \hline
\end{tabular}
\caption{Flight Duration by Payload Weight}
\label{fig:battery}
\end{table}

\subsection{Event-to-Detection Latency}
\label{sec:e2elatency}

The agility of my system is limited by the end-to-end latency of the
processing pipeline~(Figure~\ref{fig:sys-arch}).  Events closer in
time than this limit may not be resolvable.  For example, a
surveillance target with a jerky motion will be perceived as moving
more smoothly.  Large, but brief, deviations from the smoothed path
may not be detected.  The larger the end-to-end latency, the greater
the need for predictive approaches in tracking fast-moving
targets.  This, in turn, leads to greater likelihood of errors due
to mis-prediction.

The base end-to-end latency of the pipeline can by replicating the experiment described in Section~\ref{sec:streaming-experiment-setup} and Figure~\ref{fig:streaming-experiment}. The
drone is kept stationary in a lab setting, with its camera pointing at a display attached to the cloudlet. Except
for the fact that the drone is not flying, everything else (hardware, software, and network) is identical to Figure~\ref{fig:sys-arch}.  The cloudlet-connected display shows the current time at millisecond
granularity.  An image of this timestamp is captured by the drone's camera, transmitted downstream, and recovered at the end of the pipeline.  Its difference from current time at recovery gives the end-to-end latency.  Figure~\ref{fig:e2elatency} presents my results from 30 samples. The distribution is heavy-tailed, with a mean of 1138~ms and a standard deviation of 157~ms.  The high mean and variability arise from jitter in LTE transmission, as well as from processing and scheduling delays on the drone, watch, and cloudlet.


\begin{figure}
\centering
\includegraphics[width=0.7\linewidth]{chapter4/FIGS/new_mtp_watch.png}
\caption{Distribution of Detection Latency (Galaxy Watch)}
\label{fig:e2elatency}
\end{figure}


\subsection{Task-1: Object Detection While Hovering}
\label{sec:task1}

\subsubsection{Task Description}
\label{sec:task1-desc}

The accuracy of the computer vision pipeline complements its speed.
\begingroup
\setlength{\columnsep}{4pt}
Both are important for active vision.  A simple test is the detection
of a target on the ground when it moves into the camera's FOV.  The
problem is harder from higher altitude because objects are smaller and
DNNs perform poorly on objects that are just a few pixels in
size~\cite{Huang2017}.
Figure~\ref{fig:robomaster} shows the DJI Robomaster S1
robot~\cite{Robomaster2022} used as the detection target in my
experiments.  It is roughly the size of a small dog, and can be
remote-controlled over WiFi by a human driver.  It can also be
programmed to follow a predefined route, with speed variation in
different route segments. 

\begin{figure}
\centering
\includegraphics[width=0.2\linewidth]{chapter4/FIGS/robomaster.jpg}\\
{\footnotesize 432~x~330~x~304~mm\\[-0.05in]
\noindent(17~x~13~x~12~in)}
\caption{COTS Target}
\label{fig:robomaster}
\end{figure}

For Task-1, the robot is manually operated
on a freeform path that overlaps the FOV of the drone that is hovering
at fixed altitude.  In postprocessing, I compare ground truth~(GT) on
each processed frame with the output of the processing pipeline.  The
object detection DNN was created via transfer learning from
SSD-ResNet50~\cite{SSDResnet50} pre-trained on the COCO dataset.  The
training set was created from drone-captured images of the target
shown in Figure~\ref{fig:robomaster}.

\endgroup


\subsubsection{Results}
\label{sec:task1-results}

I perform this experiment at altitudes of 5~m, 10~m, and 15~m.
Accuracy is high at 5~m; a typical result is
Figure~\ref{fig:task1-images}(a), where the bounding box indicates
detection followed by correct target classification at high
confidence~(0.98).  The person at the top right and the distractor
object at the top middle are correctly ignored.  At 10~m, accuracy is
slightly lower.  An example of an erroneous result at 10~m is
Figure~\ref{fig:task1-images}(b), which shows a true positive~(TP)
(the target) at confidence 0.95 at bottom center, and a false
positive~(FP) (a person misclassified as the target at confidence
0.82) at center left.  At 15~m, accuracy suffers significantly.  An
example of an error at 15~m is Figure~\ref{fig:task1-images}(c).  This
shows an FP at center right (a person misclassified as the
target at confidence 0.86), and also a false negative~(FN) (missed target)
at center left.  Altitudes of 15~m and higher are clearly challenging
for this combination of target size, optical system, and processing pipeline.

\begin{figure}
\centering\small
\centering
\includegraphics[width=0.6\linewidth]{chapter4/FIGS/fig-static-detection-example2-5m.jpg}\\
(a) Altitude = 5m\\[0.1in]
\includegraphics[width=0.6\linewidth]{chapter4/FIGS/fig-static-detection-example2-10m.jpg}\\
(b) Altitude = 10m\\[0.1in]
\includegraphics[width=0.6\linewidth]{chapter4/FIGS/fig-static-detection-example2-15m.jpg}\\
(c) Altitude = 15m
\caption{Task-1 Images}
\label{fig:task1-images}
\end{figure}

Table~\ref{tab:task1-results}~(a) shows the confusion matrix for
Task-1 at a confidence threshold of 0.7.  The scoring of images used
in this matrix requires explanation. Classic measures of precision and
recall address scene classification, where an entire image is
correctly or incorrectly classified.  In contrast, my setting
involves object detection.  It is possible for a single image to have
both a FP and a TP or FN.  Figure~\ref{fig:task1-images}(a) contains
a single TP, and no errors.  This is scored as a TP in the confusion
matrix.  Figure~\ref{fig:task1-images}(b) contains both a TP and an
FP; this is scored as an FP since errors trump correctness.  When
there are multiple errors, the worst error determines the score.
Figure~\ref{fig:task1-images}(c), for example, contains both an FP and
an FN.  I view FNs (hurting recall) as more serious errors than FPs
(hurting precision), and therefore score the whole image as an FN.
These rules preserve the invariant
$$GT_P + GT_N = TP + TN + FP + FN$$
where $GT_P$ and $GT_N$ refer to ground truth positives and negatives,
and TN refers to true negatives (no target in image).

\begin{table}
\centering\small
\begin{minipage}[b]{2.4in}
\centering\small
\begin{tabular}{|c|c|c|c|c|}
\hline
Alti-&\multicolumn{2}{c|}{Ground}&\multicolumn{2}{c|}{Detected}\\
\cline{4-5}
tude&\multicolumn{2}{c|}{Truth}& Pos& Neg.\\
\hline
5 m & Pos. & 85 & 71 & 13\\
    & Neg. &  0 & 1  & 0\\
\hline
10 m & Pos. & 90 & 55 & 30\\
     & Neg. &  0 & 5  & 0 \\
\hline
15 m & Pos. & 85 & 24 & 48 \\
     & Neg. &  0 & 13 & 0\\
\hline
\end{tabular}\\[0.05in]
{\footnotesize Threshold = 0.7}\\
(a) Confusion Matrix\\
\end{minipage}
\begin{minipage}[b]{1.4in}
\centering
\begin{tabular}{|c|c|c|}
\hline
Alti-& Prec- & Re-\\
tude & ision & call\\
\hline
5 m & 0.99 & 0.85 \\
10 m &0.92 & 0.65\\
15 m & 0.65 & 0.33 \\
\hline
\end{tabular}\\[0.05in]
(b) Precision and Recall
\end{minipage}

\caption{Task-1 Results}
\label{tab:task1-results}
\vspace{-0.2in}
\end{table}


At 5~m, a total of 85 frames are processed.  The column labeled
``Ground Truth'' shows that all 85 contain a target instance.  The
``Detected'' columns show that 71 out of the 85 are correctly detected
and classified~(TPs), but 13 are missed~(FNs).  At 10~m, all 90
processed frames contain an instance of the target, but only 55 of
them are correctly detected~(TPs).  There are 30~FNs and 5~FPs.  At
15~m, accuracy suffers considerably.  Out of 85 total processed
frames, all are GT-positive.  However, only 24 of them are correctly
detected~(TPs).  There are 48~FNs and 13~FPs.
Table~\ref{tab:task1-results}~(b) shows the precision and recall
resulting from this confusion matrix.  These values are excellent at
5~m.  Recall is noticeably degraded at 10~m.  Both precision and
recall suffer at 15~m.  These results suggest the importance of active
vision. Dropping to a lower altitude could confirm
or refute the sighting of an object from higher altitude. 

\subsection{Task-2: Keeping Sight of  a Moving Object}
\label{sec:task2}

\subsubsection{Task Description}
\label{sec:task2-desc}

Processing a frame in Task-1 does not lead to actuation of the drone.
In contrast, Task-2 represents a simple form of active vision. After
detecting a moving target, the drone yaws to keep the target visible
in the frame. There is no forward or backward motion, only rotation to
keep the object in the FOV.  Target speed and motion predictability
clearly influence this task.  A fast-moving target that unpredictably
and frequently changes its path is clearly hard to track.  As
discussed earlier~(\S~\ref{sec:e2elatency}), the end-to-end latency of
processing constrains tracking agility. With the help of the pilot
(using the FreeFlight app), the Anafi Ai can also perform this task
and thus I use it as a benchmark for my platform.

\begingroup
\setlength{\columnsep}{4pt}
\begin{figure}
\centering
\includegraphics[width=0.4\linewidth]{chapter4/FIGS/fig-yaw-ushape.pdf}\\
\caption{Target Occlusion}
\label{fig:yaw}
\end{figure}

As shown in Figure~\ref{fig:yaw}, I set up a rectangular
(approximately 20~m~x~15~m) course marked by 4 cones.  The drone is
placed near the center of the rectangle and takes off to a fixed
altitude of 2 meters. A 2~m tall by 0.5~m wide foam pillar is placed
to occlude the target along the center of the back edge. For each run,
the target moves along the right, back, and left edges in a u-shape
two times. To test the ability of each platform to reacquire lost
targets, I vary the time spent being occluded, starting with no
pause, then pausing for 2 seconds behind the pillar, and finally a 5
second pause. Three runs of each delay were recorded.  For the Anafi
Ai, the pilot must draw a bounding box around the target in order to
start tracking; my platform automatically acquires and re-acquires
the target.  Since my platform is constrained to below 1~FPS, we
compare results from the two platforms on frames that are one second
apart.  The Anafi Ai captures video at 30~FPS, so I expect its
responsiveness in this task to be significantly better.

\begin{table}
	\centering\small
	\begin{tabular}{|c|c|c|c|c|c|c|}
		\hline
		 &  &  & \multicolumn{2}{c|}{Success} & Slow & \\
		Delay & Run & Total  & \multicolumn{2}{c|}{\footnotesize (Target Present)} & Act- &  Fail\\
		\cline{5-5} 
		(s)&  &       Frames  &         & $\rm \frac{Present}{Total}$ & uation  & \\ 
		\hline
		& 1 & 63 & 60 & & 3 & 0 \\
		0 & 2 & 62 & 62 & 91.2\% \scriptsize{(11.4\%)}  & 0 & 0 \\
		& 3 & 60 & 47 & & 0 & 13\\
		\hline
		& 1 & 58 & 56 & & 2 & 0 \\
		2 & 2 & 58 & 57 & 98.3\% \scriptsize{(1.7\%)} & 1 & 0 \\
		& 3 & 64 & 64 & & 0 & 0 \\
		\hline
		& 1 & 59 & 53 & & 0 & 6 \\
		5 & 2 & 53 & 53 & 90.3\% \scriptsize{(9.4\%)} & 0 & 0  \\
		& 3 & 53 & 43 & & 0 & 10 \\
		\hline
	\end{tabular}
	\begin{captext}
		\centering \\[0.1cm] \small Figures in parentheses are standard deviations. \\
	\end{captext}
{(a) my Platform}\\[0.2in]

\vspace{0.2in}

\begin{tabular}{|c|c|c|c|c|c|c|}
\hline
		 &  &  & \multicolumn{2}{c|}{Success} & Slow & \\
Delay & Run & Total  & \multicolumn{2}{c|}{\footnotesize (Target Present)} & Act- &  Fail\\
\cline{5-5} 
(s)&  &       Frames  &         & $\rm \frac{Present}{Total}$ & uation  & \\ 
\hline
    & 1 & 76 & 76 &    & 0 & 0 \\
0 & 2 & 78 & 78 & 100\% \scriptsize{(0\%)} & 0 & 0 \\
    & 3 & 80 & 80 &    & 0 & 0 \\
\hline
    & 1 & 79 & 79 &        & 0 & 0 \\
2 & 2 & 80 & 80 & 100\% \scriptsize{(0\%)} & 0 &  0 \\
    & 3 & 86 & 86 &        & 0 &  0 \\
\hline
    & 1 & 78 & 32 &        & 0 &  46 \\
5 & 2 & 77 & 29 & 39.4\% \scriptsize{(1.7\%)} & 0 & 48 \\
    & 3 & 81 & 32 &        & 0 &  49 \\
\hline
\end{tabular}
\begin{captext}
\centering \\[0.1cm] \small Figures in parentheses are standard deviations. \\
\end{captext}
{(b) Anafi Ai}

\caption{Task-2 Results}
\label{tab:task2-results}
\end{table}


\subsubsection{Results}
\label{sec:task2-results}
For successful tracking, both sensing and actuation are important.  If
the drone is sluggish in executing a yaw command, even perfect
processing may not keep the target in the FOV at all times.
Four outcomes are possible for each processed frame:
\begin{itemize}
	\item{the target is visible in the frame ({\small ``Success''}).}
	
	\item{the target is missing in this frame because of slow actuation, but present in the next ({\small ``Slow Actuation''}).}
	
	\item{the target is missing both in this frame and the next.
		This is scored as a tracking failure ({\small ``Fail''}).}
	
	\item{the target is occluded.
		This is not included in the frame total and is omitted from the results.}
\end{itemize}


Table~\ref{tab:task2-results}(a) and (b) compare the results for my
platform and the Anafi Ai.  At 0~s occlusion, my platform performs
well, but run 3 experiences some tracking failures.  My system's low
framerate stream accounts for these failures.  None of the frames
received by the cloudlet included the target as it traversed the last
corner of the pattern.  In contrast, the Anafi Ai experiences no
failures at 0~s occlusion.  At 2~s occlusion, both my platform and
the Anafi Ai perform very well.  However, my platform experiences a
few instances of slow actuation.  Both platforms are able to reliably
reacquire the target as it reappears from behind the obstacle.  At 5~s
occlusion, the limitations of the Anafi Ai are exposed. Such a long
period of occlusion causes the Ai's optical flow tracking algorithm to
become confused, often mistaking the pillar or background objects for
the target. my platform's DNN tracking handles the increased
occlusion well, with only a modest increase in the number of failures.

\begin{figure}
\begin{minipage}[b]{0.5\linewidth}
\centering
\includegraphics[width=0.6\linewidth]{chapter4/FIGS/fig-pattern-square.pdf}\\
{(a) Square}\\
\end{minipage}
\begin{minipage}[b]{0.5\linewidth}
\centering
\includegraphics[width=0.6\linewidth]{chapter4/FIGS/fig-pattern-cross.pdf}\\
{(b) Cross}\\
\end{minipage}
\caption{Tracking Patterns}
\label{fig:patterns}
\end{figure}

\subsection{Task-3: Following at a Fixed Leash Distance}
\label{sec:task3}

\subsubsection{Task Description}
\label{sec:task3-desc}

Only limited actuation is needed to keep the target in the
drone's FOV in Task-2.  More substantial actuation is required for
Task-3.  After detecting a moving target, the drone moves to keep it
at a preset {\em leash distance.} This compounds latency issues as the
drone must now yaw and reposition itself correctly when the target
maneuvers quickly. At high target speeds (over 2.5~m/s), this can
prove difficult because even small actuation mistakes can result in
total loss of visual contact. Although the Anafi Ai can statically
track moving objects, it does not have a 
following feature which makes a direct comparison of its performance infeasible.

I programmed the target to move at a constant speed over flat ground
in a specified pattern.  I used speeds of 1.5~m/s (slow), 2.5~m/s
(medium), and 3.5~m/s (fast).  These speeds roughly correspond to a
person walking, jogging slowly, and running.  Two patterns were used:
a square of side 10~m~(Figure~\ref{fig:patterns}(a)), and a cross with
four arms of 5~m each~(Figure~\ref{fig:patterns}(b)).  The square
embodies abrupt change of trajectory after 10~m of straight line travel,
while the change of trajectory in the cross occurs after only 5~m.

As in Task-1, the drone is initially hovering at fixed altitude.  I used altitudes of 5~m and 10~m in my experiments, but omitted 15~m
since Section~\ref{sec:task1-results} indicates poor performance at this altitude. Once the target is detected, the drone tracks it with a
preset gimbal pitch and leash distance designed to adequately frame
the target. If the drone loses sight of the target, it hovers until
the object moves back into its FOV. No active search is currently made
by the drone to reacquire a lost target.

\subsubsection{Results}
\label{sec:task3-results}

I use the same scoring rubric of ``Success,'' ``Slow Actuation,'' and
``Fail'' as for Task-2.  At a target speed of 1.5~m/s, the results for
all runs in Table~\ref{tab:task3-results-5m-square} confirm
successful tracking with only occasional failure.  When speed
increases to 2.5~m/s, and then to 3.5~m/s, the number of failures
increases sharply.  This is consistent with the computer vision
processing pipeline following real-world scene changes too slowly, due
to the very low frame rate~(0.7~FPS).  I show
later~(\S\ref{sec:discussion-results}) that increasing frame rate
improves tracking.  The effects of anomalously high LTE latency are
visible in Run 3.  This points to the challenge of using public
cellular networks which can experience unpredictable changes in bandwidth and latency.

\begin{table}
	\centering\small
	\begin{tabular}{|c|c|c|c|c|c|c|}
		\hline
		Speed & Run & Total & \multicolumn{2}{c|}{Success} & Slow & Fail\\
		(m/s) &  & Frames  & \multicolumn{2}{c|}{\footnotesize (Target Present)} & Act- &  \\
		\cline{5-5} 
		&  &         &         & $\rm \frac{Present}{Total}$ & uation  & \\ 
		\hline
		& 1 & 82 & 80 & & 1 & 1 \\
		1.5 & 2 & 85 & 76 & 95.2\% \scriptsize{(5\%)}  & 0 & 9 \\
		& 3 & 77 & 76 & & 1 & 0\\
		\hline
		& 1 & 82 & 49 & & 2 & 31 \\
		2.5 & 2 & 84 & 50 & 55\% \scriptsize{(8\%)} & 4 & 30 \\
		& \cellcolor{red!30}3 & \cellcolor{red!30}83 & \cellcolor{red!30}38 & & \cellcolor{red!30}0 & \cellcolor{red!30}45 \\
		\hline
		& 1 & 87 & 46 & & 2 & 39 \\
		3.5 & 2 & 84 & 60 & 62.7\% \scriptsize{(9.3\%)} & 1 & 23  \\
		& 3 & 83 & 53 & & 4 & 26 \\
		\hline
	\end{tabular}
	\begin{captext}
		\centering \\[0.1cm] Figures in parentheses are standard deviations.\\
		Abnormally high LTE latency was observed during the highlighted run.
	\end{captext}
	\caption{Task-3 Results {\small (Altitude = 5~m, Pattern = Square)}}
	\label{tab:task3-results-5m-square}
\end{table}


Table~\ref{tab:task3-results-5m-cross} presents Task-3 results when
the pattern used is a cross rather than a square.  Comparing the
``Fail'' columns of Tables~~\ref{tab:task3-results-5m-square} and
\ref{tab:task3-results-5m-cross}, there is a noticeable decrease in
failures at speeds of 2.5~m/s and 3.5~m/s when the pattern is a cross.
These results are consistent with the cross being less demanding than
the square for tracking.

\begin{table}
\centering\small
\begin{tabular}{|c|c|c|c|c|c|c|}
\hline
Speed & Run & Total & \multicolumn{2}{c|}{Success} & Slow & Fail\\
(m/s) &  & Frames  & \multicolumn{2}{c|}{\footnotesize (Target Present)}& Act-  &  \\
\cline{5-5} 
      &  &         &         & $\rm \frac{Present}{Total}$ & uation  & \\ 
\hline
    & 1 & 83 & 83 &    & 0 & 0 \\
1.5 & 2 & 88 & 88 & 100\% \scriptsize{(0\%)} & 0 & 0 \\
    & 3 & 78 & 78 &    & 0 & 0 \\
\hline
    & 1 & 85 & 67 &        & 0 & 18 \\
2.5 & 2 & 80 & 75 & 88.6\% \scriptsize{(8.5\%)} & 0 &  5 \\
    & 3 & 88 & 82 &        & 1 &  5 \\
\hline
    & 1 & 82 & 82 &        & 0 &  0 \\
3.5 & 2 & 85 & 59 & 89\% \scriptsize{(17\%)} & 1 & 25 \\
    & 3 & 86 & 84 &        & 2 &  0 \\
\hline
\end{tabular}
\begin{captext}
\centering \\[0.1cm] Figures in parentheses are standard deviations. \\
\end{captext}
\caption{Task-3 Results {\footnotesize (Altitude = 5~m, Pattern = Cross)}}
\label{tab:task3-results-5m-cross}
\end{table}

\begin{table}
	\centering\small
	\begin{tabular}{|c|c|c|c|c|c|c|}
		\hline
		Speed & Run & Total & \multicolumn{2}{c|}{Success} & Slow & Fail\\
		(m/s) &  & Frames  & \multicolumn{2}{c|}{\footnotesize (Target Present)}& Act-  &  \\
		\cline{5-5} 
		&  &         &         & $\rm \frac{Present}{Total}$ & uation  & \\ 
		\hline
		& 1 & 83 & 65 & & 1 & 17 \\
		1.5 & 2 & 78 & 74 & 81.6\% \scriptsize{(12\%)}  & 0 & 4 \\
		& 3 & 88 & 63 & & 3 & 19\\
		\hline
		& 1 & 82 & 52 & & 2 & 39 \\
		2.5 & 2 & 83 & 48 & 62.3\% \scriptsize{(4\%)} & 1 & 34 \\
		& 3 & 87 & 57 & & 0 & 30 \\
		\hline
		& 1 & 88 & 57 & & 1 & 30 \\
		3.5 & 2 & 83 & 45 & 64.8\% \scriptsize{(10.5\%)} & 1 & 37  \\
		& 3 & 89 & 67 & & 3 & 19 \\
		\hline
	\end{tabular}
	\begin{captext}
		\centering \\[0.1cm] Figures in parentheses are standard deviations. \\
	\end{captext}
	\caption{Task-3 Results {\footnotesize (Altitude = 10~m, Pattern = Square)}}
	\label{tab:task3-results-10m-square}
\end{table}


At an altitude of 10~m, the drone's FOV is increased, but there is a
significant drop in precision and recall as shown in
Table~\ref{tab:task1-results}~(b).  This leads to an increase in the
number of tracking failures relative to 5~m, regardless of the target
pattern or speed.  The effect is most apparent at the slowest speed:
the results for 1.5~m/s in Table~\ref{tab:task3-results-10m-square}
show higher failures than the results at 1.5~m/s in
Table~\ref{tab:task3-results-5m-square}.  Similarly, the results for
1.5~m/s in Table~\ref{tab:task3-results-10m-cross} show higher
failures than the results for 1.5~m/s in
Table~\ref{tab:task3-results-5m-cross}. These effects persist at
higher speeds, but are less obvious.  The improvement at
2.5~m/s between Tables~\ref{tab:task3-results-5m-square} and
\ref{tab:task3-results-10m-square} is due to the high-latency LTE
anomaly mentioned earlier.

\begin{table}
	\centering\small
	\begin{tabular}{|c|c|c|c|c|c|c|}
		\hline
		Speed & Run & Total & \multicolumn{2}{c|}{Success} & Slow & Fail\\
		(m/s) &  & Frames  & \multicolumn{2}{c|}{\footnotesize (Target Present)}& Act-  &  \\
		\cline{5-5} 
		&  &         &         & $\rm \frac{Present}{Total}$ & uation  & \\ 
		\hline
		& 1 & 80 & 75 &    & 0 & 5 \\
		1.5 & 2 & 85 & 85 & 87.3\% \scriptsize{(16.8\%)} & 0 & 0 \\
		& 3 & 85 & 58 &    & 0 & 27 \\
		\hline
		& 1 & 84 & 67 &        & 0 & 17 \\
		2.5 & 2 & 81 & 81 & 86.6\% \scriptsize{(11.6\%)} & 0 &  0 \\
		& 3 & 85 & 68 &        & 0 &  17 \\
		\hline
		& 1 & 86 & 62 &        & 0 &  24 \\
		3.5 & 2 & 86 & 85 & 82.9\% \scriptsize{(14.1\%)} & 1 & 0 \\
		& 3 & 86 & 67 &        & 1 &  18 \\
		\hline
	\end{tabular}
	\begin{captext}
		\centering \\[0.1cm] Figures in parentheses are standard deviations. \\
	\end{captext}
	\caption{Task-3 Results {\footnotesize (Altitude = 10~m, Pattern = Cross)}}
	\label{tab:task3-results-10m-cross}
\end{table}

\subsection{Task-4: Close Inspection}
\label{sec:task4}

\subsubsection{Task Description}
\label{sec:task4-desc}

Task-4 corresponds to the classic active vision tactic of ``taking a
closer look'' that was mentioned earlier~(\S\ref{sec:introduction}).   It
begins with the drone hovering at 15~m.  As in Task-1, the target
moves in a freeform path that is manually controlled over WiFi.  When
the target is detected in the FOV of the drone, confirmation at the
lower altitude of 5~m is attempted.  During the descent, the target is
kept in the FOV using yaw and gimbal actuation; the drone's
pitch and roll are not modified.  If multiple targets are detected at
15~m, only confirmation of the highest-confidence detection is
attempted.



\subsubsection{Results}
\label{sec:task4-results}

The results for Task-4 are shown in Table~\ref{tab:task4-results}.
The row labeled ``Static 15~m'' correspond to the 15~m results from
Table~\ref{tab:task1-results}~(b).  Relative to that baseline, both
precision and recall improve by nearly 30\% by ``taking a closer
look.''  There is, of course, a cost in time because actuation
involves physical motion of the drone to the lower altitude.  Further,
the increased FOV at higher altitude offers wider coverage.  For these
reasons, better accuracy at higher altitude will always be valuable.
However, when such improvement is not possible due to limitations of
the drone's optical system or processing pipeline, autonomously
descending to a lower altitude for target confirmation can be
effective.

\begin{table}
	\centering\small
		\centering\small
			\begin{tabular}{|c|c|c|}
			\hline
			Altitude& Precision & Recall\\
			\hline
			Close Inspection (15m-5m) & 0.94 & 0.68 \\
			Static 15 m (\S\ref{sec:task1-results}) & 0.65 & 0.33 \\
			\hline
		\end{tabular}\\
		\caption{Task-4 Results}
		\label{tab:task4-results}
\end{table}

\subsection{Task-5: Obstacle Avoidance}
\label{sec:task5}

\subsubsection{Task Description}
\label{sec:task5-desc}

\begin{figure}
\vspace{0.1in}
\begin{minipage}[b]{0.495\linewidth}
\centering
\includegraphics[width=0.95\linewidth]{chapter4/FIGS/fig-bridge-raw.jpg}\\
{\small (a) Raw Input}\\
\end{minipage}
\begin{minipage}[b]{0.495\linewidth}
\centering
\includegraphics[width=0.95\linewidth]{chapter4/FIGS/fig-bridge-midas.jpg}\\
{\small (b) Output of MiDaS}\\
\end{minipage}
\caption{Bridge Obstacle Avoidance}
\label{fig:midas-sample}
\end{figure}


Many commercial drones have on-board obstacle avoidance capabilities,
typically based on stereoscopic camera depth inference.
The Parrot Anafi Ai has such capability, and can move out of the way
of obstacles when the drone is moving forwards or backwards. my
platform only has a single RGB camera, and must therefore rely upon
monocular depth estimation.  I use a DNN-based algorithm called MiDaS
~\cite{Ranftl2022} to provide relative depth estimates.  Using MiDaS
on each frame received by the cloudlet, I construct an inverse
relative depth map.  Based on the rate of change of relative depth
across frames, I identify obstacles in the flight path and actuate
away from them.  Figure~\ref{fig:midas-sample}(a) shows an input frame
from one of my flights, as the drone approaches a pillar on a bridge.
Figure~\ref{fig:midas-sample}(b) shows the depth-encoded output of my
algorithm on this frame.  The drone actuates towards the green dot
labeled ``safe'' to avoid the pillar on the left.

Task-5 compares my platform's monocular obstacle avoidance, using my
visual pipeline at 0.7~FPS, with the stereoscopic obstacle avoidance
of the Anafi Ai using on-board computing at 30~FPS.  I place the 2~m
tall by 0.5~m wide foam pillar used in Task-2~(\S\ref{sec:task2})
directly in the drone's path. The drone is instructed to fly at 1
m/s at a fixed altitude of 1~m directly towards the obstacle.  I
capture a trace of the drone's flight path across 3 different runs.

\begin{figure}
\begin{minipage}[b]{0.495\linewidth}
\centering
\includegraphics[width=0.95\linewidth]{chapter4/FIGS/fig-obstacle-path-ours.png}
{(a) my Platform}
\end{minipage}
\begin{minipage}[b]{0.495\linewidth}
\centering
\includegraphics[width=0.95\linewidth]{chapter4/FIGS/fig-obstacle-path-anafiai.png}
{(b) Anafi Ai}
\end{minipage}
\caption{Task-5 Results}
\label{fig:obstacle-path}
\end{figure}

\subsubsection{Results}
\label{sec:task5-results}

Figure~\ref{fig:obstacle-path}(a) plots the flight path of each run
for my platform, along with the position of the pillar.
Figure~\ref{fig:obstacle-path}(b) plots the flight paths of the Anafi
Ai on the same task.  Both platforms successfully avoid the obstacle
in all cases, but do so using very different tactics. The
low frame rate and high end-to-end latency of my pipeline forces my
platform to be very conservative, and to give the obstacle a wide
berth.  Well past the obstacle, the drone has not yet returned to its
original flight path.  In contrast, the stereoscopic cameras, high
frame rate and low end-to-end processing latency of the Anafi Ai
together enable it to be much less conservative in obstacle avoidance.
The flight paths cluster more tightly around the obstacle, and the drone
soon returns to its original flight path.

\subsection{Summary of Results}
\label{sec:take-away}

My evaluation began with a single top-level question.  Could
autonomous active vision be successfully implemented on a drone with
severe constraints on both local compute and offloading?  The maximum
offloading throughput of 0.7~FPS imposed by LTE thermal constraints on
the watch~(\S\ref{sec:stream-in-practice}) is a severe bottleneck.  The
heavy-tailed distribution of end-to-end processing pipeline latency,
with a mean of over 1~s~(\S\ref{sec:e2elatency}), is another
bottleneck.  Combined with limitations of onboard processing on the
watch and the quality of the drone's optical system, these constraints
pose a formidable barrier.

In spite of this barrier, the results presented~(\S\ref{sec:task1} to
\S\ref{sec:task5}) show that active vision capabilities such as
tracking a moving object and confirming object detection by dropping
to lower altitude are feasible at credible target speeds.  Even with
the current implementation, one can perform useful tasks involving
active vision.  The range of feasible tasks can be broadened via
hardware advancements in the drone and the watch, combined with more
sophisticated algorithms that can then be supported.  Progress on this
front would also enable the superior resources of the cloudlet to be better
utilized.  Right now, the benefit of the cloudlet is being muted by the low
sustainable offloading rate. Even so, thanks to the modular design of SteelEagle, it is still possible to port new AI algorithms with minimum effort into the pipeline which could perform better on this limited stream.

\subsection{Benefit of Increased Frame Rate}
\label{sec:discussion-results}

The results for Task-3~(\S\ref{sec:task3}) make it clear that the
drone's frame rate of 0.7~FPS is a major limiting factor for tracking.
When tracking high speed objects that perform erratic maneuvers, the
drone often only has one or two detections to actuate upon. It often
does not get feedback on actuation errors until the target has exited
the frame entirely. Once that happens, tracking fails and the target
is lost.

To quantify the benefits of a higher frame rate, I repeated the most
difficult combination of speed and pattern in my Task-3 experiments:
3.5~m/s for a square pattern. Instead of using the watch, I used a
ground-based laptop to play exactly the same role.  The laptop
connects to the drone over WiFi, and connects to the cloudlet via a
commercial cellular LTE network.  In every respect other than the fact
that the laptop is not flying with the drone, the processing
pipeline is unchanged from  Figure~\ref{fig:sys-arch}.
The tracking algorithm is also unchanged from that described for
Task-3~(\S~\ref{sec:task3}). Although the laptop experiences the same
WiFi, RTSP video stream, and LTE conditions as the watch did, it does
not suffer from the same processing or LTE thermal limitations.  This
enables offloading of video processing to the cloudlet a much higher frame
rate. I chose a figure of 3~FPS since I believe that this is
realistically achievable with slightly better watch hardware. Even the
Samsung Galaxy 4 watch can sustain 3~FPS for over two minutes if it is
pre-cooled with an ice pack.  This is the full length of one run of
Task-3~(\S~\ref{sec:task3}).

\begin{table}
        \centering\small
        \begin{tabular}{|c|c|c|c|c|c|c|}
                \hline
                Frame & Run & Total & \multicolumn{2}{c|}{Success} & Slow & Fail\\
                Rate  &  & Frames  & \multicolumn{2}{c|}{{\footnotesize (Target Present)}}& Act-  &  \\
                \cline{5-5}
                (FPS) &  &         &         & $\rm \frac{Present}{Total}$  & uation  & \\
                \hline
                & 1 & 361 & 283 &        & 1 &  77 \\
                3& 2 & 360 & 342 & 83.7\% \scriptsize{(9.8\%)} & 1 & 17 \\
                & 3 & 361 & 280 &        & 0 &  82 \\
                \hline
                    & 1 & 87 & 46 & & 2 & 39 \\
                0.7 & 2 & 84 & 60 & 62.7\% \scriptsize{(9.3\%)} & 1 & 23  \\
                & 3 & 83 & 53 & & 4 & 26 \\
                \hline
        \end{tabular}
        \begin{captext}
                \centering \\[0.1cm] Speed was 3.5~m/s.  Figures in parentheses are standard deviations.
        \end{captext}
        \caption{Increased Frame Rate {\footnotesize (Altitude = 5~m, Pattern = Square)}}
        \label{tab:taskfps-results}
\end{table}

\begin{table}
        \centering\small
        \begin{tabular}{|c|c|c|}
                \hline
                Platform & Weight (g) & Throughput (FPS)\\
                \hline
                Galaxy Watch 4 & 26 & 1 \\
                Future Offload Device & $<$50 & 3\\
                Pixel 4a & 143 & 5 \\
                Dell Latitude 5420 Laptop & 2500+ & 6 \\
                \hline
        \end{tabular}\\
        \caption{Throughput and Weight by Platform}
        \label{tab:throughput}
\end{table}

\begin{table}
        \centering
        \begin{tabular}{|c|c|c|c|}
                \hline
                & Inference & Effective Throughput \\
                & (ms) & (FPS) \\
                \hline
                Object & 56.3 & 17.8  \\
                Detection & (5.9) &  \\
                \hline
                Obstacle & 124.3 & 8.1 \\
                Avoidance & (4.9) &  \\
                \hline
        \end{tabular}
        \begin{captext}
                \centering \\[0.1cm] Standard deviation in parentheses. \\
        \end{captext}
        \caption{Cloudlet Performance}
        \label{tab:cloudlet-perf}
\end{table}

Table~\ref{tab:taskfps-results} presents my results for an altitude of
5~m.  For easy comparison, the 3.5~m/s results from
Table~\ref{tab:task3-results-5m-square} are reproduced below the new
results.  Increasing the frame rate from 0.7~FPS to 3~FPS greatly
improves tracking --- almost a 20\% increase in the ``Success''
column.  Even the worst run at 3~FPS achieves 77.5\% success which is
6.1\% better than the best run at 0.7~FPS at 71.4\%
success.  This improvement is obtained without modifying my tracking
algorithm. 

Since frame rate is such an important factor in successful tracking,
it is natural to speculate on what might be possible with future
hardware advancements.  For example, if a future offload device was no heavier
than it is today but had the processing power of today's smartphones,
how much better could it do tracking?  To gain some insight into these
speculative questions, I tested the processing pipeline of
Figure~\ref{fig:sys-arch} using different offload platforms.  Experiments were performed with an identical setup as in Section~\ref{sec:e2elatency}.

Table~\ref{tab:throughput} shows the throughput in FPS and the weight
of various computing platforms today. The first two rows are the
current platform and a theoretical future offload device.  The third row is a Pixel 4a smartphone, which is able to sustain 5~FPS. At 143~g, it is too heavy
for my drone to carry, but its 5x improvement in throughput is very
attractive for robust tracking. The fourth row is a Dell Latitude 5420
laptop, which is able to sustain 6 FPS; clearly, its 2.5~kg weight is
far beyond the payload lift of any ultra-light drone.  Since the
laptop can decode video and transmit it over WiFi at 30~FPS, the
bottleneck shifts to the LTE link and the cloudlet's DNN inference
time.  As Table~\ref{tab:cloudlet-perf} shows, the cloudlet is able
to inference at roughly 17.8~FPS for object detection~(Task-1 to
Task-4) and 8.1~FPS on obstacle avoidance~(Task-5). Thus, an improved offload device on the drone could take better advantage of cloudlet resources.




